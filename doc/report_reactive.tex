\documentclass[11pt]{article}

\usepackage{amsmath}
\usepackage{textcomp}
\usepackage[top=0.8in, bottom=0.8in, left=0.8in, right=0.8in]{geometry}
% add other packages here

% put your group number and names in the author field
\title{\bf Exercise 2: A Reactive Agent for the Pickup and Delivery Problem}
\author{Group 39: Hugo Bonnome, Pedro Amorim}

% the report should not be longer than 3 pages

\begin{document}
\maketitle

\section{Problem Representation}

\subsection{Representation Description}
% describe how you design the state representation, the possible actions, the
% reward table and the probability transition table
The topology $T$ is graph defined by $T = \{C, P\}$ where $C$ is the set of cities
in the topology and $P$ the set of paths connecting these cities.
\subsubsection{State representation}
The state $s$ of a given agent is defined by $s = \{c, t_d, N_c\}$ where $c \in
C$ is the city where the agent currently is, $t_d \in C \cup \{None\}$ indicates
whether there is a task to city $d$ in $c$ (being equal to $None$ when no task
is available) and $N_c \subseteq C$ is the set of cities that can be reached
from $c$, in other words the neighbours of $c$.

\subsubsection{Actions}
The agent can:
\begin{itemize}
\item Move towards a neighbour $n$, this will be denoted $M(n)$

\item Pickup a task in the current city and deliver it to the destination city,
  this will be denoted $D(t_d)$. We assume that the agent never attempts the
  pickup action if there is no task available in its current city.
\end{itemize}

\subsubsection{Reward}
For the action of moving to a neighbour:
$$R(\{c, t_d, N_c\}, M(n)) = -dist(c, n)$$
where $n \in N_c$ and $dist(c,n)$ is the shortest path distance between $c$ and
$n$. This value can be justified by the fact every km that we travel without a
profit implies a loss.

For the action of picking up a task and delivering it:
$$R(\{c, t_d, N_c\}, D(t_d)) = AR(c, d) \frac{1}{dist(c, d)}$$
with $AR(c, d)$ being the average reward from delivering a task from city
$c$ to city $d$ which is ponderated by the distance between both cities.

\subsubsection{Probability transition table}
The uncertainty in the world state only comes from the presence of a task in a
given city or not. It does not depend on the type of action taken by the agent.

$$ p(\{c, t, N_c\}, (M(n)|P(n)), \{n, t_d, N_n\}) = P(n,d)$$
$$ p(\{c, t, N_c\}, (M(n)|P(n)), \{n, None, N_n\}) = probNoTask(n)$$
where $P(n,d)$ is the probability of there being a task in city $n$ whose
destination is $d$ and $probNoTask(n)$ is the probability of city $n$ having no
task which can be computed by $1 - \Sigma_{c \in C} P(n, c)$.

\subsection{Implementation Details}

_______________________ ACTIONTYPE

an ENUM with two Plans : MOVE or PICKUP

_______________________ STATE

Define a city where the state belong and a taskDestination that's represents
a package to deliver (it can be null if their is no task in this city).
So the state is represented by the both attributes : CITY + TASK.

It implements :
-getAvailableAction() : return a list of ActionState (an edge to another city with an ActionType),
i.e all decisions the agent could make in a given state.
-probability() : return the probability to be in the state knowing that we are in the city of the state
i.e : proba(this.taskDestination | this.city)


_______________________ ACTION EDGE

Define an action that can be taken by an agent in a state so it's instanciated with :
fromState + ActionType(MOVE/PICKUP) + CityDestination.

It implements :
- getImmediateReward() : return the immediate reward associated with the action, calculated as follow :
MOVE : -this.fromState.getCity().distanceTo(this.getDestination())
PICKUP : this.taskDistribution.reward(fromState.getCity(), getDestination())/ this.fromState.getCity().distanceTo(this.getDestination())


_______________________ GRAPH

It generates all possibles states and provides the following functions :
getStateIt() : iterate over all states
getStateIt(city) : iterate over all states in a given city


_______________________ STATE & GRAPH (continued)

-STATE contains two more attributes :
bestAction : the most rewarded ActionState (initially null)
evReward : the average expected reward of the best action (initially 0)

-GRAPH contains a method :
expectedRewardIn(city) : return the sum of each state's evReward multiplied by the state probability in the given city
i.e : what will be the average evReward if i go to city and if i play perfectly in this city

- STATE contains a method :
update(discount, graph) : update evReward and bestAction by reevaluate all their available actions.
for each availableAction take max :
reward = availableAction.getImmediateReward() + discount*graph.expectedRewardIn(availableAction.getDestination());
i.e : immediate reward of the action + discount*weighted sum of state.reward

- GRAPH contains a method :
update(discount) : call for each state in the graph state.update(discount)


to make each evReward/bestAction converge to their respective values we just have to call graph.update(discount)
until done.


\section{Results}
% in this section, you describe several results from the experiments with your reactive agent


\subsection{Experiment 1: Discount factor}
Discount factor allow MDP to weight the future reward, the more the discount factor is closed to one, the more
MDP will be efficient to gain long-term expected value.
Before all, we think that the discount factor in this game will not have a strong influence on the average reward,
because the rules of the world are enough simplistic to require to look far away in the future.

\subsubsection{Setting}
% you describe how you perform the experiment (you also need to specify the configuration used for the experiment)
We run two experiments, the first one with a discount factor equals to 0.85 and the second one at 0.15.
\subsubsection{Observations}

#REACTIVE df : 0.85 : The total profit after 16386 actions is 613281328 (average profit: 37427.152935432685)
#REACTIVE df : 0.15 : The total profit after 16744 actions is 625253670 (average profit: 37341.95353559484)

As expected the discount factor does not produce subtantial variation on the average reward,
so we've got 37341 for a discount factor at 0.15 vs 37427 for a discount factor at 0.85 : it makes only about 0.2\% improvement.
However it remains a relevant parameter in more complexe game like chess when one action could provide
the best reward in short-term but lead in a dead-end-street in the long-term.

\subsection{Experiment 2: Comparisons with the given dummy agent}

\subsubsection{Setting}

Let's test our agent against the dummy one that follow a basic strategy :
1. move to a random neighbor if their is no task available or with 15\% probability if a task is provided.
2. carry out the package otherwise.

Reactive agent : MDP based action.
Dummy agent : given in the skeleton

\subsubsection{Observations}
After about 19'000 actions we've got the following result :
#DUMMY : The total profit after 18843 actions is 593389062 (average profit: 31491.220187868174)
#REACTIVE : The total profit after 19769 actions is 738573948 (average profit: 37360.207800091055)

Reactive agent increase his average reward by around 18\% in comparison to the dummy agent.

\subsection{Experiment 3: Comparisons with modified dummy agent}

\subsubsection{Setting}

By intuition we tend to says that a good strategy is to always carry out a package when the state
allows it. That's why we modified the dummy agent to always accept a work and compare the result
with the MDP reactive one.

Reactive agent : MDP based action.
Dummy agent : pPickup is increase to 1.0 (still random when no task is provided)

\subsubsection{Observations}
After about 17'000 actions we've got the following result:
#DUMMY : The total profit after 16522 actions is 612041360 (average profit: 37044.023725941166)
#REACTIVE : The total profit after 18550 actions is 696576186 (average profit: 37551.276873315364)

Reactive agent increase his average reward by about 1\% in comparison to the dummy agent.

This small difference comes from the fact that all cities has the same uniform distribution of task
and because deliver a package is ~always the best action.

The slight difference can be explained from the fact that our reactive agent improve his score
by choosing better action when no work is offered in a city, in this case it will
perfectly choose the good tradeof between taking the most closest city (low immediate cost)
and the most centered city (best reward in the future).

\end{document}