\documentclass[11pt]{article}

\usepackage{amsmath}
\usepackage{textcomp}
\usepackage[top=0.8in, bottom=0.8in, left=0.8in, right=0.8in]{geometry}
% add other packages here

% put your group number and names in the author field
\title{\bf Exercise 2: A Reactive Agent for the Pickup and Delivery Problem}
\author{Group 39: Hugo Bonnome, Pedro Amorim}

% the report should not be longer than 3 pages

\begin{document}
\maketitle

\section{Problem Representation}

\subsection{Representation Description}
% describe how you design the state representation, the possible actions, the
% reward table and the probability transition table
The topology $T$ is graph defined by $T = \{C, P\}$ where $C$ is the set of cities
in the topology and $P$ the set of paths connecting these cities.
\subsubsection{State representation}
The state $s$ of a given agent is defined by $s = \{c, t_d, N_c\}$ where $c \in
C$ is the city where the agent currently is, $t_d \in C \cup \{None\}$ indicates
whether there is a task to city $d$ in $c$ (being equal to $None$ when no task
is available) and $N_c \subseteq C$ is the set of cities that can be reached
from $c$, in other words the neighbours of $c$.

\subsubsection{Actions}
The agent can:
\begin{itemize}
\item Move towards a neighbour $n$, this will be denoted $M(n)$

\item Pickup a task in the current city and deliver it to the destination city,
  this will be denoted $D(t_d)$. We assume that the agent never attempts the
  pickup action if there is no task available in its current city.
\end{itemize}

\subsubsection{Reward}
For the action of moving to a neighbour:
$$R(\{c, t_d, N_c\}, M(n)) = -dist(c, n)$$
where $n \in N_c$ and $dist(c,n)$ is the shortest path distance between $c$ and
$n$. This value can be justified by the fact every km that we travel without a
profit implies a loss.

For the action of picking up a task and delivering it:
$$R(\{c, t_d, N_c\}, D(t_d)) = AR(c, d) \frac{1}{dist(c, d)}$$
with $AR(c, d)$ being the average reward from delivering a task from city
$c$ to city $d$ which is ponderated by the distance between both cities.

\subsubsection{Probability transition table}
The action of moving towards a neighbour $n$:
$$ p(\{c, t, N_c\}, M(n), \{n, t_d, N_n\}) = P(n,d)$$
$$ p(\{c, t, N_c\}, M(n), \{n, None, N_n\}) = 1 - P(n,d)$$
where $P(n,d)$ is the probability of there being a task in city $n$ whose
destination is $d$.

The action of picking up and delivering a task:
$$ p(\{c, t_d, N_c\}, P(t_d), \{d, t_i, N_d\}) = P(d, i) $$
$$ p(\{c, t_d, N_c\}, P(t_d), \{d, None, N_d\}) = 1 - P(d, i) $$
where $i$ is the destination of a potential task in city $d$.
\subsection{Implementation Details}
% describe the implementation details of the representations above and the implementation details of the reinforcement learning algorithm you implemented

\section{Results}
% in this section, you describe several results from the experiments with your reactive agent

\subsection{Experiment 1: Discount factor}
% the purpose of this experiment is to understand how the discount factor influences the result

\subsubsection{Setting}
% you describe how you perform the experiment (you also need to specify the configuration used for the experiment)

\subsubsection{Observations}
% you describe the experimental results and the conclusions you inferred from these results

\subsection{Experiment 2: Comparisons with dummy agents}
% you compare the results of your agent with two dummy agents: the random agent that was already given in the starter files and another dummy agent that you define and create. You should report the results from the simulations using the topologies given in the starter files and optionally, additional topologies that you create.

\subsubsection{Setting}
% you describe how you perform the experiment and you describe the dummy agent you created (you also need to specify the configuration used for the experiment)

\subsubsection{Observations}
% elaborate on the observed results

\vdots

\subsection{Experiment n}
% other experiments you would like to present

\subsubsection{Setting}

\subsubsection{Observations}

\end{document}