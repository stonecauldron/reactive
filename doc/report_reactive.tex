\documentclass[11pt]{article}

\usepackage{amsmath}
\usepackage{textcomp}
\usepackage[top=0.8in, bottom=0.8in, left=0.8in, right=0.8in]{geometry}
% add other packages here

% put your group number and names in the author field
\title{\bf Exercise 2: A Reactive Agent for the Pickup and Delivery Problem}
\author{Group 39: Hugo Bonnome, Pedro Amorim}

% the report should not be longer than 3 pages

\begin{document}
\maketitle

\section{Problem Representation}

\subsection{Representation Description}
% describe how you design the state representation, the possible actions, the
% reward table and the probability transition table
The topology $T$ is graph defined by $T = \{C, P\}$ where $C$ is the set of cities
in the topology and $P$ the set of paths connecting these cities.
\subsubsection{State representation}
The state $s$ of a given agent is defined by $s = \{c, t_d, N_c\}$ where $c \in
C$ is the city where the agent currently is, $t_d \in C \cup \{None\}$ indicates
whether there is a task to city $d$ in $c$ (being equal to $None$ when no task
is available) and $N_c \subseteq C$ is the set of cities that can be reached
from $c$, in other words the neighbours of $c$.

\subsubsection{Actions}
The agent can:
\begin{itemize}
\item Move towards a neighbour $n$, this will be denoted $M(n)$

\item Pickup a task in the current city and deliver it to the destination city,
  this will be denoted $D(t_d)$. We assume that the agent never attempts the
  pickup action if there is no task available in its current city.
\end{itemize}

\subsubsection{Reward}
For the action of moving to a neighbour:
$$R(\{c, t_d, N_c\}, M(n)) = -dist(c, n)$$
where $n \in N_c$ and $dist(c,n)$ is the shortest path distance between $c$ and
$n$. This value can be justified by the fact every km that we travel without a
profit implies a loss.

For the action of picking up a task and delivering it:
$$R(\{c, t_d, N_c\}, D(t_d)) = AR(c, d) \frac{1}{dist(c, d)}$$
with $AR(c, d)$ being the average reward from delivering a task from city
$c$ to city $d$ which is weighted by the distance between both cities.

\subsubsection{Probability transition table}
The uncertainty in the world state only comes from the presence of a task in a
given city or not. It does not depend on the type of action taken by the agent.

$$ p(\{c, t, N_c\}, (M(n)|P(n)), \{n, t_d, N_n\}) = P(n,d)$$
$$ p(\{c, t, N_c\}, (M(n)|P(n)), \{n, None, N_n\}) = probNoTask(n)$$
where $P(n,d)$ is the probability of there being a task in city $n$ whose
destination is $d$ and $probNoTask(n)$ is the probability of city $n$ having no
task which can be computed by $1 - \Sigma_{c \in C} P(n, c)$.

\subsection{Implementation Details}

\subsubsection{ActionType}
an ENUM with two Plans : MOVE or PICKUP

\subsubsection{State}
Define a city where the state belongs and a taskDestination that represents a
package to deliver (it can be null if there is no task in this city). So the
state is represented the both attributes : CITY + TASK.

It implements :
\begin{itemize}
\item getAvailableAction() : returns a list of ActionState (an edge to another
city with an ActionType), i.e all decisions the agent could make in a given
state.
\item probability() : returns the probability to be in the state knowing that we
are in the city of the state i.e : probability(this.taskDestination | this.city)
\end{itemize}


\subsubsection{ActionEdge}
Define an action that can be taken by an agent in a state so it's instantiated
with fromState + ActionType(MOVE/PICKUP) + CityDestination.

It implements :
\begin{itemize}
\item getImmediateReward() : return the immediate reward associated with the
action, calculated as follow :
\item MOVE : -this.fromState.getCity().distanceTo(this.getDestination())
\item PICKUP : this.taskDistribution.reward(fromState.getCity(),
getDestination())/ this.fromState.getCity().distanceTo(this.getDestination())
\end{itemize}

\subsubsection{Graph}
It generates all possibles states and provides the following functions :
\begin{itemize}
\item getStateIt() : iterate over all states
\item getStateIt(city) : iterate over all states in a given city
\end{itemize}

\subsubsection{State \& Graph}

STATE contains two more attributes :
\begin{itemize}
\item bestAction : the most rewarded ActionState (initially null)
\item evReward : the average expected reward of the best action (initially 0)
\end{itemize}

GRAPH contains a method :
\begin{itemize}
\item expectedRewardIn(city) : returns the sum of each state's evReward
multiplied by the state probability in the given city i.e : what will be the
average evReward if I go to this city and if I act perfectly in this city.
\end{itemize}

STATE contains a method :
\begin{itemize}
\item update(discount, graph) : update evReward and bestAction by reevaluate all
their available actions. for each availableAction take max : reward =
availableAction.getImmediateReward() +
discount*graph.expectedRewardIn(availableAction.getDestination()); i.e :
immediate reward of the action + discount*weighted sum of state.reward
\end{itemize}

GRAPH contains a method :
\begin{itemize}
\item update(discount) : call for each state in the graph state.update(discount)
\end{itemize}

To make each evReward/bestAction converge to their respective values we just
have to call graph.update(discount) until the differences between each update
are lower than a preset threshold.

\section{Results}
% in this section, you describe several results from the experiments with your reactive agent


\subsection{Experiment 1: Discount factor}
Discount factor allow MDP to weight the future reward, the closer the discount
factor is to one, the more efficient MDP will be in gaining long-term expected
value. A priori, we think that the discount factor in this setting will not
have a strong influence on the average reward, because the rules of the world
are simplistic enough to not require long-term consideration of future states.

\subsubsection{Setting}
% you describe how you perform the experiment (you also need to specify the configuration used for the experiment)
We ran two experiments, the first one with a discount factor equal to 0.85 and
the second one at 0.15.
\subsubsection{Observations}

\begin{itemize}
\item REACTIVE df : 0.85 : The total profit after 16386 actions is 613281328
(average profit: 37427.152935432685)
\item REACTIVE df : 0.15 : The total profit after 16744 actions is 625253670
(average profit: 37341.95353559484)
\end{itemize}
As expected the discount factor does not produce substantial variation on the
average reward, so we have 37341 for a discount factor at 0.15 vs 37427 for a
discount factor at 0.85 : it only makes about 0.2\% improvement. However it
should be a more relevant parameter in more complex games like chess when one
action could provide the best short-term reward but lead to a dead-end further
down the road.

\subsection{Experiment 2: Comparisons with the given dummy agent}

\subsubsection{Setting}

Let's test our agent against the dummy one that follows a basic strategy :

\begin{itemize}
\item Move to a random neighbor if there is no task available or with 15\%
probability if a task is provided.
\item Deliver the package otherwise.
\end{itemize}

Reactive agent : MDP based action.
Dummy agent : Random agent given in the skeleton.

\subsubsection{Observations}
After about 19'000 actions we have the following result:
\begin{itemize}
\item DUMMY : The total profit after 18843 actions is 593389062 (average profit:
31491.220187868174)
\item REACTIVE : The total profit after 19769 actions is 738573948 (average
profit: 37360.207800091055)
\end{itemize}

The reactive agent has an average reward that is about 18\% higher in comparison
to the dummy agent.

\subsection{Experiment 3: Comparisons with modified dummy agent}

\subsubsection{Setting}
An intuitive strategy would be to always deliver a package when the state allows
it. This is the reason we modified the dummy agent to always accept a task when
available and compared the result with the MDP reactive agent.

Reactive agent : MDP based action.
Dummy agent : pPickup was increased to 1.0 (random movement when no task is
available)

\subsubsection{Observations}
After about 17'000 actions we have the following results:
\begin{itemize}
\item DUMMY : The total profit after 16522 actions is 612041360 (average profit:
37044.023725941166)
\item REACTIVE : The total profit after 18550 actions is 696576186 (average profit:
37551.276873315364)
\end{itemize}

The reactive agent has an average reward that is about 1\% higher in comparison
to the dummy agent

This small difference comes from the fact that all cities have the same uniform
distribution of tasks and as such delivering a package is almost always the best
action to take.

The slight difference can be explained by the fact that our reactive agent
improved his score by choosing a better action when no work is offered in a
city. In this case it will choose the best trade of between taking the
closest city (low immediate cost) or the most central city. (best reward
in the future).

\end{document}